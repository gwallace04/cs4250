\documentclass[letterpaper, 11pt]{article}
%\usepackage[hmargin = 1in, vmargin = 1in]{geometry}
\usepackage{amsmath}
\usepackage{amssymb}
\usepackage{enumitem}
\usepackage{mathrsfs}
\usepackage{tikz}
\usepackage{graphicx}
\usepackage{algorithmicx}
\usepackage{algpseudocode}
% \doublespacing
\setlength{\headheight}{14pt}
\usepackage{fancyhdr}
\pagestyle{fancy}
\rhead{Gabriel Wallace}
\lhead{Comp Sci 3130}

\newcommand{\card}{\text{Card}}
\newcommand{\N}{\mathbb{N}}
\newcommand{\R}{\mathbb{R}}
\newcommand{\Z}{\mathbb{Z}}
\newcommand{\Q}{\mathbb{Q}}

\newcommand{\inv}{^{-1}}
\newcommand{\abs}[1]{\lvert #1 \rvert}
\newcommand{\hwnumber}[1]{\medskip \noindent\textbf{#1.} \smallskip}
\newcommand{\hwnumbersec}[3]{\medskip \noindent\textbf{#1.} Chapter #2 #3
\newcommand{\ans}{\noindent\textbf{A:} }
\newline\noindent\textbf{Q:}}
\newcommand{\Mod}[1]{\ \mathrm{mod}\ #1}
\newcommand{\Alg}[1]{\medskip \noindent\textbf{ALGORITHM} \( #1 \)} 
\newcommand{\To}{\textbf{ to }}

\begin{document}
\begin{center}
	{\LARGE Homework 2}\\
\end{center}

\hwnumbersec{1}{1}{RQ \#1}
Why is it useful for a programmer to have some background in language design,
even though he or she may never actually design a programming language?

\ans The reasons given in the book are: 
\begin{itemize}
\item Increased capacity to express ideas
\item Improved background for choosing appropriate languages
\item Increased ability to learn new languages
\item Better understanding of the significance of implementation
\item Better use of languages that are already known
\item Overall advancement of computing
\end{itemize}

In essence, studying how programming languages work gives the programmer more
knowledge, which in turn makes for a better programmer. When all you have is a
hammer, everything looks like a nail. By studying language design, programmers
are able to expand their toolkit. Sometimes, a hammer is the perfect tool for
the job, but it oftentimes isn't. Furthermore, computer science and programming
are a relatively new discipline, meaning that change happens all the time. The
hammer (i.e the programming language or data structure) you are using today
could very well be obsolete by tomorrow. This is less of a problem if (a) you
have many other tools to use and (b) understand the fundamentals of language
design and thus can pick up the new tools instead.  


\end{document}
